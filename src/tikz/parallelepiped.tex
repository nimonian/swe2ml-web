\documentclass[tikz,border=5pt]{standalone}
\usepackage{tikz-3dplot}
\usepackage{swe2ml}

\begin{document}
% View: z up, x right, y into the page
\tdplotsetmaincoords{70}{120}

\begin{tikzpicture}[tdplot_main_coords, scale=5]

% Define vectors a, b, c
\coordinate (O) at (0,0,0);
\coordinate (A) at (1,0.1,0);       % Vector a
\coordinate (B) at (0,1,0.3);     % Vector b (into the page, slight z)
\coordinate (C) at (0.2,0.2,1);       % Vector c (upward)

% Build parallelepiped
\coordinate (AB) at ($(A)+(B)$);
\coordinate (AC) at ($(A)+(C)$);
\coordinate (BC) at ($(B)+(C)$);
\coordinate (ABC) at ($(A)+(B)+(C)$);

% Draw axes
\draw[->, thick] (O) -- (1.5,0,0) node[anchor=south east]{$x$};
\draw[->, thick] (O) -- (0,1.5,0) node[anchor=south west]{$y$};
\draw[->, thick] (O) -- (0,0,1.5) node[anchor=south]{$z$};

% Draw vectors a, b, c
\draw[vp-vector, c-indigo-1] (O) -- (A) node[midway, below] {$\vec{a}$};
\draw[vp-vector, c-red-1] (O) -- (B) node[midway, left] {$\vec{b}$};
\draw[vp-vector, c-green-1] (O) -- (C) node[midway, right] {$\vec{c}$};

% Draw base parallelogram (a and b)
\fill[gray!20, opacity=0.5] (O) -- (A) -- (AB) -- (B) -- cycle;

% Draw vertical edges
\draw[thick] (O) -- (C);
\draw[thick] (A) -- (AC);
\draw[thick] (B) -- (BC);
\draw[thick] (AB) -- (ABC);

% Draw top parallelogram
\fill[blue!20, opacity=0.3] (C) -- (AC) -- (ABC) -- (BC) -- cycle;

% Optional: outline the sides
\draw[thick, dashed] (C) -- (BC);
\draw[thick, dashed] (AC) -- (ABC);

\end{tikzpicture}
\end{document}
