\documentclass[tikz,border=5pt]{standalone}
\usepackage{tikz-3dplot}
\usepackage{swe2ml}

\usetikzlibrary{angles, quotes} % add this to your preamble


\begin{document}
% View: z up, x right, y into the page
\tdplotsetmaincoords{72}{25}

\begin{tikzpicture}[tdplot_main_coords, scale=4.5]
  
% Define the z-axis direction as a point
\coordinate (Z) at (0,0,1);

% Define vectors a, b, c
\coordinate (O) at (0,0,0);
\coordinate (A) at (1,0,0);       % Vector a
\coordinate (B) at (0.35,1,0);     % Vector b (into the page, slight z)
\coordinate (C) at (0.3,0.3,1);       % Vector c (upward)

% Build parallelepiped
\coordinate (AB) at ($(A)+(B)$);
\coordinate (AC) at ($(A)+(C)$);
\coordinate (BC) at ($(B)+(C)$);
\coordinate (ABC) at ($(A)+(B)+(C)$);

\filldraw[c-indigo-1, opacity=0.1] (O) -- (A) -- (AB) -- (B) -- cycle;

\draw[vp-vector] (0,0,0) -- (0,0,1.4) node[anchor=south]{$\vec{v} \times \vec{w}$};

% Draw base parallelogram (a and b)
\draw[c-text-1, thick] (A) -- (AB);
\draw[c-text-1, thick, dashed] (AB) -- (B);

% Draw vectors a, b, c
\draw[vp-vector, c-indigo-1] (O) -- (A) node[pos=0.8, below] {$\vec{v}$};
\draw[vp-vector, c-red-1] (O) -- (B) node[pos=0.8, above] {$\vec{w}$};
\draw[vp-vector, c-green-1] (O) -- (C) node[pos=0.8, above left] {$\vec{u}$};

% Draw vertical edges
\draw[c-text-1, thick] (A) -- (AC);
\draw[c-text-1, thick, dashed] (B) -- (BC);
\draw[c-text-1, thick] (AB) -- (ABC);

% Draw top
\draw[c-text-1, thick] (C) -- (BC);
\draw[c-text-1, thick] (BC) -- (ABC);
\draw[c-text-1, thick] (ABC) -- (AC);
\draw[c-text-1, thick] (AC) -- (C);

% Draw the angle θ between z-axis and vector c
\draw pic[
    draw=black,
    angle radius=50,
    angle eccentricity=1.2,
    "$\varphi$"
] {angle = C--O--Z};

\draw pic[
    draw=black,
    angle radius=50,
    angle eccentricity=1.2,
    "$\theta$"
] {angle = A--O--B};

\end{tikzpicture}
\end{document}
