\documentclass[tikz,border=5pt]{standalone}
\usepackage{tikz-3dplot}
\usepackage{swe2ml}

\usetikzlibrary{angles, quotes}


\begin{document}
% View: z up, x right, y into the page
\tdplotsetmaincoords{72}{25}

\begin{tikzpicture}[tdplot_main_coords, scale=4.5]
  
% Define the z-axis direction as a point
\coordinate (Z) at (0,0,1);


% Define vectors a, b, c
\coordinate (O) at (0,0,0);
\coordinate (A) at (1,0,0);       % Vector a
\coordinate (B) at (0.35,1,0);     % Vector b (into the page, slight z)
\coordinate (C) at (0.3,0.3,1);       % Vector c (upward)

% Build parallelepiped
\coordinate (AB) at ($(A)+(B)$);
\coordinate (AC) at ($(A)+(C)$);
\coordinate (BC) at ($(B)+(C)$);
\coordinate (ABC) at ($(A)+(B)+(C)$);

\filldraw[c-indigo-1, opacity=0.1] (O) -- (A) -- (AB) -- (B) -- cycle;


% Draw base parallelogram (a and b)
\draw[vp-path, c-indigo-1] (A) -- (AB);
\draw[vp-path, c-indigo-1, dashed] (AB) -- (B);

% Draw vectors a, b, c
\draw[vp-path, c-indigo-1] (O) -- (A);
\draw[vp-path, c-indigo-1] (O) -- (B);
\draw[vp-path, c-indigo-1] (O) -- (C);

% Draw vertical edges
\draw[vp-path, c-indigo-1] (A) -- (AC);
\draw[vp-path, c-indigo-1, dashed] (B) -- (BC);
\draw[vp-path, c-indigo-1] (AB) -- (ABC);

% Draw top
\draw[vp-path, c-indigo-1] (C) -- (BC);
\draw[vp-path, c-indigo-1] (BC) -- (ABC);
\draw[vp-path, c-indigo-1] (ABC) -- (AC);
\draw[vp-path, c-indigo-1] (AC) -- (C);

\end{tikzpicture}
\end{document}
